\section{System Limitation and Future Work}
%Although \papertitle\ proves to be convenient and easy to use, the cost for the required materials is high. For a bottle of Silver Nano Particle Ink (100 ml) the sale price is 300 USD. Additional costs of purchasing a compatible printer, printing paper and customized \papertitle\ PCB further increase the system cost.

%Non-acrylic version
%Tightening the four screws on the corners causes the PCB to flex outwards by approximately 1 mm. The compression range of a contact spring is only 0.5 mm; therefore, additional forces must be exerted in order to mitigate the outward flex and ensure the reliability for electrical connection. Future hardware designers should evenly distribute the force across the entire board via an additional rigid support structure such as acrylic. Other options include exchanging the current contact spring for one with a greater compression range.

% The cost for obtaining \papertitle's required materials is high. A bottle of Silver Nano Particle Ink (100 ml)  costs 300 USD. Additional costs of a compatible printer, printing paper and customized PCB further increase the system cost.

%The printed Silver Nano Particle Ink material is not suited for extensive prototyping due to the high resistance. Therefore until there is new technology in material sciences, this material is better suited for analog circuits.

The use of screws as the assembly mechanism produces assembly time overhead in the prototyping process. Options that can be explored are mechanisms involving magnetic adhesion, reducing the assembly overhead.

Silver nano particle ink is not suited for analog circuit prototyping due to the high resistance. Modularized analog devices communicating via I2C and SPI protocols are tentative solutions to such issue.

An issue with the current \papertitle\ prototype is that additional jumper cables are needed for connection to Arduino. In the future, modifying \papertitle\ into an Arduino compatible shield allows for complete elimination of jumper cables.

%Assembly time is an area of concern due to the need for a secure and tight assembly. The use of screws and caps (FIX NAME) requires additional time to be included in the prototyping process.