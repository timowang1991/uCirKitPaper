\section{Introduction}
% Breadboard is currently the dominant tool in any prototyping process. Despite its popular, breadboard is not without its flaws, namely the need for jumper cables which lead to cluttered and messy prototypes.

The recent development in electric circuit rapid prototyping tools, such as ink-based electrical circuitry allows users to prototype without wiring. Unfortunately, ink-based electrical circuitry is inherently difficult to modify, because electronic components are affixed on a piece of paper or board by soldering or taping. A hard-to-modify prototyping tool slows makers down due to component reattachment to a new printed circuitry. We therefore argue that the process of how circuit rapid prototyping is used for quick modification is not yet optimal.

In order to allow makers to iterate quickly, hard-to-modify techniques, such as sketching and paper prototyping, give priority to speed over functionality. This trade-off pays off in the early phases of design because it encourages the quick exploration of several versions before committing further resources, eventually leading to a better design.

We argue that the same principle should apply to circuit prototyping – a concept we call easy-to-modify circuit prototyping. In contrast to the traditional workflow, in which the circuit prototyping is always done as time-consuming soldering or hard-to-modify ink-based techniques, easy-to-modify prototyping makes all intermediate versions as fast and easy-to-modify circuits. Only at the end, when the design is finished, the complete circuit prototype is printed as PCB (Figure 2).

One ink-based circuit prototyping approach is Inkjet — a system that prints circuits speed-up by substituting wires with conductive silver ink. Unfortunately, Inkjet requires users to attach components on the circuit with hard-to-modify approaches, such as taping or soldering.

In this paper, we present a easy-to-modify approach that not only save time, but also save the cost of components. The key idea is to combine the mechanism of breadboard and ink-based circuit printing to provide an easy-to-modify and robust approach for circuit prototyping , i.e. a 3D print in which surfaces have been replaced with a wireframe mesh. Our approach runs on both standard FDM 3D printers, such as the PrintrBot or the Kossel mini (rather than on a 5 axis robot arm [12]) and Inkjet printers (Brothers MFC) users only need to install the uCirKit software. Resin-coated photo paper helps to maximize the conductivity of the printed wires.
