\section{User Study}

\subsection{Foreword}
The following section details the flow of our user study. However, a decision to remove printable circuits as a member of the three prototyping tools was made. This is a result of two aspects, reliability and functionality. 

The first reason was due to the unreliable nature of previous works in this area. In the beginning, we followed the convention laid out by Instant Inkjet Circuit \cite{Instant_Inkjet_Circuits}. Despite claiming to be an effective method, the 3M conductive tape suggested by Instant Inkjet Circuit proved to be a liability in component connection. Enormous pressure and heat had to be applied to obtain a consistent connection between component and printed circuit. Therefore, we modified our components to use IC sockets for an uniform 90 degree contact area. Thus, in our original study, users were given modified components in the printable circuit section.

However, the functionality of printable circuits was called into question. The high level of component customization required (e.g. Circuit Sticker) to allow for complete functionality or faster prototyping results in the lost of rapidness in prototyping. 

%In our pilot study, we found that the unreliable nature of printable circuits resulted in the lack of functionality in our implementation... Therefore...

Combining the two critical flaws mentioned above, we decided to exclude printable circuits from the tested tools in our study.

\subsection{Procedure}

\subsubsection{Pre-evaluation}
Users were first asked to answer a questionnaire in order to obtain basic information regarding age and gender. Then, users were asked how many times they have used breadboards in the past, and if they had any experience at all they were asked the source of their exposure. To help us pinpoint the users relative experience, we asked them what was the most difficult circuit they have prototyped. To conclude the pre-evaluation, users were asked they have tried other prototyping tools and if so, which ones.

\subsubsection{Environment Set-up}
Our study environment consists of three stages for each prototyping tool under evaluation. At each station, parts and Arduino Mega boards are all laid out for users to use. Users are shown schematics on a monitor in order to assist in completion of the tasks. At the same time, users are recorded in order to evaluate their process after the study is over.

\subsubsection{"Blink"}
Due to the fact that some users are completely foreign to prototyping and electronic circuits, we prepared a warm-up exercise obtained from the Official Arduino Website. The exercise, called "Blink", involves connecting a LED and turning it on and off every second by toggling the HIGH LOW of the LED. This warm-up activity was used before the start of every prototyping tool session in order to help users become acquainted with the tool they are using.

\subsubsection{Chosen Circuits}
In order to evaluate the three prototyping tools when handling circuits with different levels of complexity(?), we choose to construct a series of circuits, each building off the previous. The first circuit consists of a temperature sensor (LM35) and a LED that turns on whenever the temperature rises above 30 degrees Celsius. Without changing the first circuit, a RGB LED is then added to signify the range that the temperature is in. The RGB LED will change smoothly from aqua to green to red every increment of 3 degrees. Lastly, a four-digit seven segment is added to display the temperature reading in Celsius.

\subsubsection{Prototyping Tools}
Our evaluation encompasses breadboard, printable circuits, and lastly \papertitle\. As mentioned before, for every tool, users are asked to complete the warm-up exercise, then introduced to our chosen circuits. However, we choose to randomize the order in which users tried each tool, because REASON. Furthermore, during the chosen circuits part, some users are asked to complete the most complicated circuit, then work in reverse back to a simple LED with temperature sensor. We determined that this method would help REASON. 

\subsubsection{Post-Evaluation}
Users were recorded for the entirety of the time in which they were asked to perform the tasks above. After completion of the three tools, users were given a questionnaire that asked them to rank the convenience, ease of use, and modification ability of the three tools. Users were also given an opportunity to comment with any thoughts that they had during the study.