\section{Conclusion}
% In this paper, we implemented a novel rapid prototyping tool that solves the issue of cluttered wires on breadboards, inconvenient modifications on a PCP, and the single layer of PCP. \papertitle\ consists of multilayer customized PCBs with contact pads in two differing zones. Pins and contact pads on both the top layer and bottom layers have corresponding pairs in the other zone and are then electrically connected to neighboring layers through contact springs. Through the presence of contact springs on both top and bottom layers, our printed circuits are fit snugly within the layers for a secure connection. The layers are then assembled along with pieces of PCP with four screws located on each corner, completing \papertitle's structure. Results show that \papertitle\ structure leads to 43\% increase in overall performance and 78\%.

In this paper, we implemented a novel rapid prototyping tool that solves the issue of cluttered wires on breadboards, inconvenient modifications on a PCP, and the single--layer issue of PCP. \papertitle\ consists of a multilayer PCB design where the top layer possesses breadboard's fast plug-and-play mechanism and the bottom layers are of multiple PCBs each placed with a piece of PCP for fast routing connections and a solution to the single-layer issue. Results show that \papertitle\ structure leads to 43\% increase in overall performance and 78\% decrease in debug time lowering the difficulties in building complicated circuits not only for experts but beginners.

% In this paper, we proposed a novel approach to 3D proto- typing, i.e. to print wireframe representations instead of a filled model and to extrude filament directly into 3D space instead of printing layer-wise. Our validation shows that in combination, these two approaches indeed lead to a substantial speed up of up to a factor of 10 and thus allow designers to iterate more often.

% For future work, we plan to explore how fast 3D printing allows for novel types of interfaces that close the feedback loop between digital editing and physical fabrication.