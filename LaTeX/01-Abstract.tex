\begin{abstract}
% Even though considered a rapid prototyping tool, 3D printing is so slow that a reasonably sized object requires printing overnight. This slows designers down to a single iteration per day. In this paper, we propose to instead print low-fidelity wireframe previews in the early stages of the design process. Wireframe previews are 3D prints in which surfaces have been replaced with a wireframe mesh. Since wireframe previews are to scale and represent the overall shape of the 3D object, they allow users to quickly verify key aspects of their 3D design, such as the ergonomic fit.

Even though considered a prototyping tool, building a complicated circuit on a breadboard can be so slow that a fully debugged iteration requires significant time. In this paper, we propose a quick method to make prototypes in the early stages of circuit design process. Requiring no physical wires, \papertitle\ is a prototyping system in which wires and jumpers have been replaced with printed circuits and customized PCB. The results of our user study indicate a significant leap in prototyping performance using our system by X\%.

% To maximize the speed-up, we instruct 3D printers to extrude filament not layer-by-layer, but directly in 3D-space, allowing them to create the edges of the wireframe model directly one stroke at a time. This allows us to achieve speed-ups of up to a factor of 10 compared to traditional layer-based printing. We demonstrate how to achieve wireframe previews on standard FDM 3D printers, such as the PrintrBot or the Kossel mini. Users only need to install the WirePrint software, making our approach applicable to many 3D printers already in use today. Finally, wireframe previews use only a fraction of material required for a regular print, making it even more affordable to iterate.
\end{abstract}