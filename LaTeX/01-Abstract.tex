\begin{abstract}
% Even though considered a rapid prototyping tool, 3D printing is so slow that a reasonably sized object requires printing overnight. This slows designers down to a single iteration per day. In this paper, we propose to instead print low-fidelity wireframe previews in the early stages of the design process. Wireframe previews are 3D prints in which surfaces have been replaced with a wireframe mesh. Since wireframe previews are to scale and represent the overall shape of the 3D object, they allow users to quickly verify key aspects of their 3D design, such as the ergonomic fit.

Breadboarding is a common circuit prototyping technique. It requires manual routing of wires, which is error prone and time consuming as complexity increases. Printing circuits on paper using conductive ink provides fast wire routing, but components are difficult to attach/detach and wire crossing is not possible. This paper presents \papertitle, a novel approach that provides fast wire routing, easy attachment/detachment of components, and support for wire crossing. It uses printed circuits on one or more pieces of paper that are sandwiched between specially designed printed circuit boards (PCB) with contact springs. We conducted a 16-person user study with participants creating circuits at different complexity levels. Results show that \papertitle{} significantly improved debugging time by 78\% and overall completion time by 43\% compared to breadboards. Also, while beginners were 62\% slower than experts when using breadboard, beginners perform as fast as experts when using \papertitle{}.
%加關於Modification的敘述

% To maximize the speed-up, we instruct 3D printers to extrude filament not layer-by-layer, but directly in 3D-space, allowing them to create the edges of the wireframe model directly one stroke at a time. This allows us to achieve speed-ups of up to a factor of 10 compared to traditional layer-based printing. We demonstrate how to achieve wireframe previews on standard FDM 3D printers, such as the PrintrBot or the Kossel mini. Users only need to install the WirePrint software, making our approach applicable to many 3D printers already in use today. Finally, wireframe previews use only a fraction of material required for a regular print, making it even more affordable to iterate.
\end{abstract}